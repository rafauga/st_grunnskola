% Options for packages loaded elsewhere
\PassOptionsToPackage{unicode}{hyperref}
\PassOptionsToPackage{hyphens}{url}
\documentclass[
]{book}
\usepackage{xcolor}
\usepackage{amsmath,amssymb}
\setcounter{secnumdepth}{5}
\usepackage{iftex}
\ifPDFTeX
  \usepackage[T1]{fontenc}
  \usepackage[utf8]{inputenc}
  \usepackage{textcomp} % provide euro and other symbols
\else % if luatex or xetex
  \usepackage{unicode-math} % this also loads fontspec
  \defaultfontfeatures{Scale=MatchLowercase}
  \defaultfontfeatures[\rmfamily]{Ligatures=TeX,Scale=1}
\fi
\usepackage{lmodern}
\ifPDFTeX\else
  % xetex/luatex font selection
\fi
% Use upquote if available, for straight quotes in verbatim environments
\IfFileExists{upquote.sty}{\usepackage{upquote}}{}
\IfFileExists{microtype.sty}{% use microtype if available
  \usepackage[]{microtype}
  \UseMicrotypeSet[protrusion]{basicmath} % disable protrusion for tt fonts
}{}
\makeatletter
\@ifundefined{KOMAClassName}{% if non-KOMA class
  \IfFileExists{parskip.sty}{%
    \usepackage{parskip}
  }{% else
    \setlength{\parindent}{0pt}
    \setlength{\parskip}{6pt plus 2pt minus 1pt}}
}{% if KOMA class
  \KOMAoptions{parskip=half}}
\makeatother
\usepackage{longtable,booktabs,array}
\usepackage{calc} % for calculating minipage widths
% Correct order of tables after \paragraph or \subparagraph
\usepackage{etoolbox}
\makeatletter
\patchcmd\longtable{\par}{\if@noskipsec\mbox{}\fi\par}{}{}
\makeatother
% Allow footnotes in longtable head/foot
\IfFileExists{footnotehyper.sty}{\usepackage{footnotehyper}}{\usepackage{footnote}}
\makesavenoteenv{longtable}
\usepackage{graphicx}
\makeatletter
\newsavebox\pandoc@box
\newcommand*\pandocbounded[1]{% scales image to fit in text height/width
  \sbox\pandoc@box{#1}%
  \Gscale@div\@tempa{\textheight}{\dimexpr\ht\pandoc@box+\dp\pandoc@box\relax}%
  \Gscale@div\@tempb{\linewidth}{\wd\pandoc@box}%
  \ifdim\@tempb\p@<\@tempa\p@\let\@tempa\@tempb\fi% select the smaller of both
  \ifdim\@tempa\p@<\p@\scalebox{\@tempa}{\usebox\pandoc@box}%
  \else\usebox{\pandoc@box}%
  \fi%
}
% Set default figure placement to htbp
\def\fps@figure{htbp}
\makeatother
\setlength{\emergencystretch}{3em} % prevent overfull lines
\providecommand{\tightlist}{%
  \setlength{\itemsep}{0pt}\setlength{\parskip}{0pt}}
\usepackage[]{natbib}
\bibliographystyle{plainnat}
\usepackage{booktabs}
\usepackage{bookmark}
\IfFileExists{xurl.sty}{\usepackage{xurl}}{} % add URL line breaks if available
\urlstyle{same}
\hypersetup{
  pdftitle={Nótur um stærðfræðimenntun},
  pdfauthor={Ingólfur Gíslason},
  hidelinks,
  pdfcreator={LaTeX via pandoc}}

\title{Nótur um stærðfræðimenntun}
\author{Ingólfur Gíslason}
\date{2026-01-09}

\begin{document}
\maketitle

{
\setcounter{tocdepth}{1}
\tableofcontents
}
\chapter{Um textann}\label{umtextann}

Þessi texti er verk í vinnslu. Hann er um stærðfræðimenntun, einkum kennslu og nám (frekar en kerfis- eða félagsfræðilega sýn). Hún er hugsuð sem kennsluefni í námskeiðinu Stærðfræði í grunnskóla sem er fyrir stærðfræðikennaranema við Menntavísindasvið Háskóla Íslands. Hún gæti þó nýst nemendum í öðrum námskeiðum eða starfandi kennurum. Textinn byggir víða á bók Henri Picciotto og Robin Pemantle, \emph{There Is No One Way to Teach Math} og á að þjóna sem stuðningur og dýpkun við lestur og notkun þeirrar bókar.

Í textanum er áherslan á undirstöðuhugtök og praktískar leiðir í kennslu, auk rýni í stærðfræðilegt efni grunnskólans. Gengið er út frá því að lesendur vinni bæði saman og með kennara, en áhugasamar manneskjur utan slíks samhengis gætu ef til vill nýtt sér textann líka.

\section{Notkun bókar og tungumál}\label{notkun-buxf3kar-og-tungumuxe1l}

Ég ávarpa lesanda textans ýmist sem nemanda (hann), lesanda (hann), en stundum nota ég „þau``, „öll``, og önnur orð til að gefa til kynna að öll eru velkomin að lesa og læra af þessum texta, og persónur í verkefnatextum geta verið af ólíku kyni.

Bókina má aðlaga og nýta að vild, að hluta eða í heild, með eftirfarandi skilyrðum:

\begin{enumerate}
\def\labelenumi{\arabic{enumi}.}
\item
  Bókin sé ekki nýtt til að valda manneskjum skaða, græða peninga, eða til að stuðla að auknum ójöfnuði milli fólks.
\item
  Ef umtalsverðir hlutar eru nýttir í öðru verki sé upprunans hér getið.
\end{enumerate}

\chapter{Skilningur}\label{skilningur}

Hvað er skilningur? Hvað er skilningur í stærðfræði?

\section{Venslaskilningur og tækniskilningur}\label{venslaskilningur-og-tuxe6kniskilningur}

Við getum rifjað upp tvígreiningu Richard Skemps í venslaskilning og tækniskilning. Í örstuttu máli drögum við þessi hugtök saman:

\begin{itemize}
\tightlist
\item
  Venslaskilningur er að vita hvernig hægt er að setja nota stærðfræðihugtök og -aðferðir og vita hvers vegna þær eru viðeigandi og hvers vegna þær skila réttum niðurstöðum.
\item
  Tækniskilningur er að vita hvernig á að reikna tiltekin dæmi án þess að vita hvers vegna þau eru reiknuð á þann hátt, hvort það sé vit í reikningunum, eða hvernig aðferðin tengist öðrum aðferðum eða hugtökum.
\end{itemize}

Í bók Picciotto og Pemantle er \emph{skilningur} nokkurn veginn það sama og Skemp kallar venslaskilning.

\section{Skilningur í daglegu máli og í stærðfræði}\label{skilningur-uxed-daglegu-muxe1li-og-uxed-stuxe6ruxf0fruxe6uxf0i}

Til frekari dýptar getum við velt fyrir okkur hvað orðið skilningur eiginlega merkir. Í skáldsögu Fríðu Ísberg, \emph{Merking} (annað mikilvægt hugtak!) segir:

\begin{quote}
Íslenska orðið „skilningur`` hefur ákveðna einangrunarmerkingu; við reynum að skilja eitt frá öðru - hið ljóta frá hinu fallega, manninn frá dýrinu - til að einangra það og eyða svo því sem við viljum ekki. En latneska orðið „comprehendere`` - sem enskan notar - er andstæða þess: það þýðir að taka saman. „Com-`` er svipað forskeyti og „sam-`` á íslensku og „prehendere`` þýðir að grípa, ná tökum á einhverju. (bls. 194)
\end{quote}

Fleiri rannsakendur hafa lagt áherslu á að skilningur felist bæði í að aðskilja, greina sundur, og líka að taka saman, ná utan um, grípa um. Mjög fróðleg umfjöllun um skilning í íslensku og í sögulegu ljósi kennslubóka í stærðfræði er í grein Kristínar Bjarnadóttur (2022), Er áhersla á skilning nýmæli? \emph{Skírnir, 196}(2), 371--390.

Til að skýra betur hvað það merkir að \emph{skilja hugtak} segja Picciotto og Pemantle að það að skilja hugtak ætti yfirleitt að þýða að nemandi geti:

\begin{itemize}
\tightlist
\item
  Útskýrt það, útskýra \emph{hvers vegna} eitthvað er eins og það er, ekki bara nefna orð. Við ættum að biðja nemendur reglulega að útskýra bæði munnlega og skriflega.
\item
  Að snúa ferlum við. Þú skilur ekki dreifireglu nema þú getir þáttað líka, þú skilur ekki jöfnur nema þú getir búið til (margar) jöfnur sem hafa lausnina \(4\), og getir búið til jöfnu út frá grafi. Það að geta snúið við ferlum (reversibility) er prófsteinn á skilning, leið til að dýpka skilning, og í sumum tilvikum önnur leið að skilningi.
\item
  Svegjanleg notkun margra leiða. Að skilja jöfnur: með prófun, með gröfum, með töflum, með tækni, auk bókstafareiknings.
\item
  Að geta \emph{tengt milli ólíkra framsetninga}, til dæmis tákna (algebrustæða), töflu (gildistöflu) og grafs (teikning í hnitakerfi).
\item
  Yfirfærsla á ný samhengi. Til dæmis að tengja hlutföll við einslaga myndir, og fjarlægð milli punkta við reglu Pýþagórasar.
\item
  Að vita hvenær það á ekki við. Dæmi: ekki eru öll sambönd línuleg. Þess vegna er mikilvægt að rannsaka gagndæmi, hvenær eitthvað á ekki við.
\end{itemize}

\subsection{Verkefni}\label{verkefni}

Hugsum okkur spurninguna

\begin{quote}
Af hverju er \(2(x+3) = 2x + 6\)?
\end{quote}

Hvaða ályktanir um skilning nemanda getum dregið út frá eftirfarandi „svörum``, og hvernig gætum við brugðist við svörunum til að kanna skilninginn betur:

\begin{enumerate}
\def\labelenumi{\arabic{enumi}.}
\tightlist
\item
  „Það er út af dreifireglunni?{}``
\item
  „Til dæmis ef \(x=1\) þá þetta bara \(2(1+3)=2 \cdot 4\) og það er alveg eins hægt að skipta \(4\) niður og margfalda hvern hlut fyrir sig, \(2 \cdot 1 = 2 og 2 \cdot 3 = 6\) og leggja saman, \(2+6=8\).
\item
  „Ef ég skoða rétthyrning með eina hlið \(2\) og hina \(x+3\) og reikna flatarmálið, þá skiptir ekki máli hvort ég reikna hvern bút fyrir sig og legg saman eða allan í einu`` (með fylgir eftirfarandi teikning):
  \pandocbounded{\includegraphics[keepaspectratio]{_main_files/figure-latex/reitir-1.pdf}}
\end{enumerate}

\chapter{Talnaskyn, táknskyn, aðgerðaskyn}\label{skyn}

Orðið skyn á hér að fela í sér bæði skilning og skynjun.

\section{Talnaskyn (number sense)}\label{talnaskyn-number-sense}

Með talnaskyni er meðal annars átt við að hve miklu leyti við getum reiknað í huganum, tengt saman talnastaðreyndir sem við þekkjum og áttað okkur á stæðarsamanburði talna. Hugarreikningur - notkun þekktra staðreynda og að átta sig á samanburði talna (líka brota).

\subsection{Dæmi}\label{duxe6mi}

\begin{itemize}
\tightlist
\item
  Ef við reiknum \(13-9\) með því að hugsa \(13 = 10 + 3\) og notum okkur að \(10-9=1\) til að sjá að \(13-9 = 10 - 9 + 3 = 1 + 3 = 4\).
\item
  Ef við reiknum \(4 \cdot 13 = 2 \cdot 2 \cdot 13 = 2 \cdot 26 = 52\)
\end{itemize}

Skoðið fleiri dæmi á \href{https://skolathraedir.is/2021/04/21/eg-leysi-stundum-vandamalid-med-svona-hringjum-hugsun-barna-um-margfoldun/}{„Ég leysi stundum vandamálið með svona hringjum`` Hugsun barna um margföldun}.

\section{Táknskyn (symbol sense)}\label{tuxe1knskyn-symbol-sense}

Með táknskyni er meðal annars átt við:

\begin{enumerate}
\def\labelenumi{\alph{enumi}.}
\tightlist
\item
  Að geta valið hvort gagnlegt sé að nota tákn (eins og bókstafi fyrir breytur og óþekktar tölur) við lausn verkefnis eða ekki
\item
  Að geta túlkað tákn sem notuð eru til að lýsa reikningum eða aðstæðum
\item
  Fimi í bókstafareikningi (að umbreyta táknarunum í aðrar jafngildar táknarunur)
\item
  Val á réttum táknum fyrir tilteknar aðstæður
\end{enumerate}

\subsection{Verkefni}\label{verkefni-1}

Eftirfarandi spurningar reyna á táknskyn með ólíkum hætti. Í hverri línu er fullyrðing sem segja á hvort sé stundum, alltaf eða aldrei sönn. Við höfum áhuga á skilningi svo hér þarf að útskýra \emph{hvers vegna} í sérhverjum lið. Við viljum líka benda á undir hvaða flokk táknskyns hér að ofan hvert dæmi reynir helst á.

\begin{itemize}
\tightlist
\item
  \(x+2 < 2x\).
\item
  \(-x\) er neikvæð tala.
\item
  Það skiptir máli hvort við skrifum \(4a+10\) eða \(4b+10\).
\item
  Ef \(a\) og \(b\) eru breytur, þá er ómögulegt að \(a=b\).
\item
  \((x+5)^2 = x^2 + 5^2\).
\item
  \(-(y-1) = -y -1\) ?
\item
  \(-a^2 = (-a)^2\) ?
\item
  Ef þú margfaldar þrjár tölur sem standa saman í talnaröðinni (eins og \(13\), \(14\) og \(15\)) verður útkoman alltaf margfeldi af \(6\) (með öðrum orðum: \(6\) gengur upp í henni).
\item
  Ef \(3\) gengur upp í þversummu tölu þá gengur \(3\) gengur upp í tölunni.
\end{itemize}

Táknskyn er ekki eitthvað sem kemur fljótt eða sjálfkrafa. Það þarf að kenna nemendum að vinna með tákn.

\section{Aðgerðaskyn (operation sense)}\label{auxf0geruxf0askyn-operation-sense}

Sameiginlegt með talnaskyni og táknskyni er aðgerðaskyn. Þá er átt við að skilja reikniaðgerðir, ólíka eiginleika þeirra og tengsl þeirra á milli, auk þess að hafa vit á því að nota þær við viðeigandi aðstæður og verkefnum.

\subsection{Dæmi}\label{duxe6mi-1}

Skoðum rununa \(5, 8, 11, 14, 17...\) Hvað er í gangi hér?

\begin{itemize}
\tightlist
\item
  Endurtekin samlagning, \(5, 5+3, 5+3+3, ...\) sem hægt er að tjá með margföldun
\item
  Táknað með \(5 + 3n\) og tengist framsetningu á línu \(y = 3x+5\)
\end{itemize}

Við viljum meðal annars að nemendur skilji að endurtekna samlagningu má reikna með margföldun, og átti sig á og geti notað dreifiregluna \(a(b+c)\) í sinni talnahugsun og hugarreikningi.

\subsection{Verkefni}\label{verkefni-2}

Víxlregla um samlagningu segir að jafnan \(a+b = b+a\) sé alltaf sönn.

\begin{enumerate}
\def\labelenumi{\arabic{enumi}.}
\tightlist
\item
  Gildir víxlregla um frádrátt? Er hægt að segja eitthvað um tölurnar \(a\) og \(b\) ef \(a-b = b-a\)?
\item
  Er einhver regla á því hvað gerist ef liðum er víxlað í frádrætti? Hvað gerist?
\item
  Gildir víxlregla um margföldun?
\item
  Gildir víxlregla um deilingu?
\end{enumerate}

\bibliography{book.bib,packages.bib}

\end{document}
