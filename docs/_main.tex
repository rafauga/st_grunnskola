% Options for packages loaded elsewhere
\PassOptionsToPackage{unicode}{hyperref}
\PassOptionsToPackage{hyphens}{url}
\documentclass[
]{book}
\usepackage{xcolor}
\usepackage{amsmath,amssymb}
\setcounter{secnumdepth}{5}
\usepackage{iftex}
\ifPDFTeX
  \usepackage[T1]{fontenc}
  \usepackage[utf8]{inputenc}
  \usepackage{textcomp} % provide euro and other symbols
\else % if luatex or xetex
  \usepackage{unicode-math} % this also loads fontspec
  \defaultfontfeatures{Scale=MatchLowercase}
  \defaultfontfeatures[\rmfamily]{Ligatures=TeX,Scale=1}
\fi
\usepackage{lmodern}
\ifPDFTeX\else
  % xetex/luatex font selection
\fi
% Use upquote if available, for straight quotes in verbatim environments
\IfFileExists{upquote.sty}{\usepackage{upquote}}{}
\IfFileExists{microtype.sty}{% use microtype if available
  \usepackage[]{microtype}
  \UseMicrotypeSet[protrusion]{basicmath} % disable protrusion for tt fonts
}{}
\makeatletter
\@ifundefined{KOMAClassName}{% if non-KOMA class
  \IfFileExists{parskip.sty}{%
    \usepackage{parskip}
  }{% else
    \setlength{\parindent}{0pt}
    \setlength{\parskip}{6pt plus 2pt minus 1pt}}
}{% if KOMA class
  \KOMAoptions{parskip=half}}
\makeatother
\usepackage{longtable,booktabs,array}
\usepackage{calc} % for calculating minipage widths
% Correct order of tables after \paragraph or \subparagraph
\usepackage{etoolbox}
\makeatletter
\patchcmd\longtable{\par}{\if@noskipsec\mbox{}\fi\par}{}{}
\makeatother
% Allow footnotes in longtable head/foot
\IfFileExists{footnotehyper.sty}{\usepackage{footnotehyper}}{\usepackage{footnote}}
\makesavenoteenv{longtable}
\usepackage{graphicx}
\makeatletter
\newsavebox\pandoc@box
\newcommand*\pandocbounded[1]{% scales image to fit in text height/width
  \sbox\pandoc@box{#1}%
  \Gscale@div\@tempa{\textheight}{\dimexpr\ht\pandoc@box+\dp\pandoc@box\relax}%
  \Gscale@div\@tempb{\linewidth}{\wd\pandoc@box}%
  \ifdim\@tempb\p@<\@tempa\p@\let\@tempa\@tempb\fi% select the smaller of both
  \ifdim\@tempa\p@<\p@\scalebox{\@tempa}{\usebox\pandoc@box}%
  \else\usebox{\pandoc@box}%
  \fi%
}
% Set default figure placement to htbp
\def\fps@figure{htbp}
\makeatother
\setlength{\emergencystretch}{3em} % prevent overfull lines
\providecommand{\tightlist}{%
  \setlength{\itemsep}{0pt}\setlength{\parskip}{0pt}}
\usepackage[]{natbib}
\bibliographystyle{plainnat}
\usepackage{booktabs}
\usepackage{bookmark}
\IfFileExists{xurl.sty}{\usepackage{xurl}}{} % add URL line breaks if available
\urlstyle{same}
\hypersetup{
  pdftitle={Nótur um stærðfræðimenntun},
  pdfauthor={Ingólfur Gíslason},
  hidelinks,
  pdfcreator={LaTeX via pandoc}}

\title{Nótur um stærðfræðimenntun}
\author{Ingólfur Gíslason}
\date{2026-01-08}

\begin{document}
\maketitle

{
\setcounter{tocdepth}{1}
\tableofcontents
}
\chapter{Um textann}\label{umtextann}

Þessi texti er um stærðfræðimenntun, einkum kennslu og nám (frekar en kerfis- eða félagsfræðilega sýn). Hún er hugsuð sem kennsluefni í námskeiðinu Stærðfræði í grunnskóla sem er fyrir stærðfræðikennaranema við Menntavísindasvið Háskóla Íslands. Hún gæti þó nýst nemendum í öðrum námskeiðum eða starfandi kennurum.

Í textanum er áherslan á undirstöðuhugtök og praktískar leiðir í kennslu, auk rýni í stærðfræðilegt efni grunnskólans. Gengið er út frá því að lesendur vinni bæði saman og með kennara, en áhugasamar manneskjur utan slíks samhengis gætu ef til vill nýtt sér textann líka.

\section{Notkun bókar og tungumál}\label{notkun-buxf3kar-og-tungumuxe1l}

Ég ávarpa lesanda textans ýmist sem nemanda (hann), lesanda (hann), en stundum nota ég „þau``, „öll``, og önnur orð til að gefa til kynna að öll eru velkomin að lesa og læra af þessum texta, og persónur í verkefnatextum geta verið af ólíku kyni.

Bókina má aðlaga og nýta að vild, að hluta eða í heild, með eftirfarandi skilyrðum:

\begin{enumerate}
\def\labelenumi{\arabic{enumi}.}
\item
  Bókin sé ekki nýtt til að valda manneskjum skaða, græða peninga, eða til að stuðla að auknum ójöfnuði milli fólks.
\item
  Ef umtalsverðir hlutar eru nýttir í öðru verki sé upprunans hér getið.
\end{enumerate}

\chapter{Talnaskyn, táknskyn, aðgerðaskyn}\label{skyn}

\section{Talnaskyn (number sense)}\label{talnaskyn-number-sense}

Hugarreikningur - notkun þekktra staðreynda og að átta sig á samanburði talna (líka brota).

Táknskyn (symbol sense)
* Að geta valið hvort gagnlegt sé að nota tákn (eins og bókstafi fyrir breytur og óþekktar tölur) við lausn verkefnis eða ekki
* Að geta túlkað tákn sem notuð eru til að lýsa reikningum eða aðstæðum
* Fimi í bókstafareikningi (að umbreyta táknarunum í aðrar jafngildar táknarunur)
* Val á réttum táknum fyrir tilteknar aðstæður

\subsection{Dæmi:}\label{duxe6mi}

\begin{itemize}
\tightlist
\item
  Er það satt að ef þú margfaldar þrjár tölur sem standa saman í talnaröðinni (eins og 13, 14 og 15) verði útkoman alltaf margfeldi af 6 (með öðrum orðum: 6 gengur upp í henni).
\item
  Er það satt að 3 gangi upp í tölu ef 3 gengur upp í þversummu hennar?
\item
  Er \((x+5)^2 = x^2 + 5^2\) ?
\item
  Er \(-(y-1) = -y -1\) ?
\item
  Er \(-a^2 = (-a)^2\) ?
\item
  Skiptir máli hvort við skrifum \(4a+10\) eða \(4b+10\) ?
\item
  Ef við segjum að \(a\) og \(b\) séu tölur, getur þá verið að \(a=b\) ?
\end{itemize}

Táknskyn er ekki eitthvað sem kemur fljótt eða sjálfkrafa. Það þarf að kenna nemendum að vinna með tákn.

Sameiginlegt með talnaskyni og táknskyni er aðgerðaskyn.

\subsection{\texorpdfstring{Dæmi: \(5, 8, 11, 14, 17...\) Hvað er í gangi hér?}{Dæmi: 5, 8, 11, 14, 17... Hvað er í gangi hér?}}\label{duxe6mi-5-8-11-14-17...-hvauxf0-er-uxed-gangi-huxe9r}

\begin{itemize}
\tightlist
\item
  Endurtekin samlagning, 5, 5+3, 5+3+3, \ldots{} sem hægt er að tjá með margföldun
\item
  Táknað með \(5 + 3n\) og tengist framsetningu á línu \(y = 3x+5\)
\end{itemize}

\bibliography{book.bib,packages.bib}

\end{document}
